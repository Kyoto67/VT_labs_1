\setcounter{page}{6}

\includegraphics[width=1\textwidth]{Screenshot 2021-11-24 221446.png}


\begin{tabular}{p{0.5\linewidth} p{0.5\linewidth}}
~~~~В поисках новых источников сырья люди все чаще обращаются к несметным богатствам кладовых полезных ископаемых, скрытых водной гладью морей и океанов. По оценкам ученых размеры запасов нефти, металлических руд и других минералов, находящихся на дне и под дном водныхпространств, в несколько раз превосходят запасы этих ископаемых,имеющиеся на суше.

~~~~Но как узнать, где под водой скрыты эти сокровища? Для этого надо знать геологическое строение земной коры покрытых водой районов. По картам геологического строения специалисты могут судить, где и какие полезные ископаемые следует искать.

~~~~Для геологических исследований морского дна, как и на суше, применяются геофизические методы разведки — сейсморазведка и бурение скважин. Но для получения подробных сведений о геологии района нужно наугад бурить очень много скважин, каждую из которых непросто пробурить даже на суше. А бурение скважины в морском дне является еще более громоздким и чрезвычайно дорогостоящим делом.

~~~~Сейсморазведка исследует геологическое строение земной коры с помощью взрывов. В земле бурят небольшую скважину и в ней взрывают заряд. Образовавшиеся при взрыве расширяющиеся газы вызывают деформацию почвы, сжимая слой час-& тиц, непосредственно окружающий место взрыва. Возникшие в сжатом слое силы упругости передают давление соседним слоям, и от точки взрыва по породе во все стороны распространяется упругая ударная волна (рис. 1)

~~~~Предположим, что под первым слоем породы I на глубине  залегает слой II, сложенный из породы другой плотности. Глубину залегания слоя II определить нетрудно

(см. рис. 1):

\begin{displaymath}S=\sqrt{S_1^2 - \frac{(S_2)^2}{2}} ,\end{displaymath}

где $S_1 = \frac{1}{2}ABC$ - путь,  пройденный
прямой волной до слоя II (такой же путь проходит отраженная волна), $S_2$ — расстояние по поверхности между точкой взрыва и сейсмоприемником, регистрирующим момент прихода отраженной волны.

~~~~Энергия упругих волн при прохождении по земным породам частично поглощается. Доля поглощенной энергии зависит от частоты колебаний волны. При взрыве образуются волны инфразвуковых (ниже 16 герц) и звуковых (от 16 герц до 20000 герц) частот. Волны этих частот сравнительно мало поглощаются земными породами, и этим определяется их использование в сейсморазведке.

~~~~Сейсморазведка успешно применяется для изучения строения мор-
\end{tabular}

\newpage
\setcounter{page}{51}
\begin{center}
 \begin{tabular}{|c|c|c|c|c|c|c|}
\hline    
Фигура& \rotatebox{90}{Контроль} &\rotatebox{90}{Ферзь} & \rotatebox{90}{Ладья} & \rotatebox{90}{Слон} &\rotatebox{90}{Конь} & \rotatebox{90}{Пешка} \\  \hline 
Число внешней устройчивости & & & & & &  \\ 
устройчивости & 9 & 5 & 8 & 8 & 12 & 32  \\ [3 mm] \hline
Число внутренней  & & & & & & \\ 
устойчивости & 16 & 8& 8& 14& 32&32  \\ [3 mm] \hline     
     
\end{tabular}   
\end{center}

\begin{multicols*}{3}
ходящих позиций - 92.
Особенно легко наши задачи решаются для ладьи - наверное, проявляется ее прямолинейность. Ладей-часовых надо иметь
\vspace{5mm}
\includegraphics[width=0.3\textwidth]{ris6.png}
Рис. 6. \\
8, их достаточно расположить вдоль любой горизонтали или вертикали. Нетрудно также показать, что и число внутренней устойчивости для ладьи равно 8, а соответствующих расположений имеется $8!$ - число перестановок из 8 элементов*) ($n!=1\cdot	2\cdot	...\cdot n$). Позиция на рисунке 6 удовлетворяет обоим требованиям. Докажите, что если ладьи пренумерованы, то есть отличаются друг от друга, то необходимых расположений 

-------- \\
\tiny *) См., например, статью Н.Я. Виленкина "Комбинаторика" ("Квант" № 1, 1971). 

\columnbreak

\small имеется уже $(8!)^2$. На доске размером $n\times n$ эти числа соотвественно равны $n!$ и $(n!)^2$.
Слонов, чтобы справиться со всей доской, достаточно иметь столько же, сколько и ладей - 8 (рис. 7). Однако разместить слонов на доске так, чтобы они не угрожали друг другу, можно почти вдвое больше, чем ладей - 14 (рис. 8). Для доказательства рассмотрим диагонали, идущие вверх слева направо (на рисунке 8 они пунктирные), всего таких диагоналей 15. На каждой из них может стоять не более одного слона, так как слоны не должны угрожать друг другу. Следовательно, общее число слонов не превышает 15. Но ровно 15 слонов также нельзя расставить: крайние из наших диагоналей состоят из одного-единственного поля. Таким образом, искомое число не более 14. Одно из решений приведено на рисунке 8. Аналогичные рассуждения показывают, что на доске размером $n\times n$ можно расставить не больше $2n-2$ слонов.
\columnbreak

Докажите, что число допустимых расстановок слонов на произвольной
\includegraphics[width=0.3\textwidth]{ris7.png}
Рис. 7.

\begin{flushleft}
\includegraphics[width=0.3\textwidth]{ris8.png}
\end{flushleft}
Рис. 8. \\
доске равно квадрату некоторого числа.

Теперь перейдем к коням. Поставив 32 коня на одни белые или одни черные поля, мы найдем расположение, при котором они не угрожают друг другу. Покажем, что 32 - число внутренней устойчивости для коей. Для этого разобьем доску на 8 равных прямоугольников (рис. 9). Легко заметить, что конь, помещенный в
\end{multicols*}