%! Author = kyoto
%! Date = 08.03.2022


\section{Задание}
\textbf{Доработать программу из лабораторной работы №6 следующим образом:}
\begin{enumerate}
    \item Организовать хранение коллекции в реляционной СУБД (PostgresQL). Убрать хранение коллекции в файле.
    \item Для генерации поля id использовать средства базы данных (sequence).
    \item Обновлять состояние коллекции в памяти только при успешном добавлении объекта в БД
    \item Все команды получения данных должны работать с коллекцией в памяти, а не в БД
    \item Организовать возможность регистрации и авторизации пользователей. У пользователя есть возможность указать пароль.
    \item Пароли при хранении хэшировать алгоритмом MD5
    \item Запретить выполнение команд не авторизованным пользователям.
    \item При хранении объектов сохранять информацию о пользователе, который создал этот объект.
    \item Пользователи должны иметь возможность просмотра всех объектов коллекции, но модифицировать могут только принадлежащие им.
    \item Для идентификации пользователя отправлять логин и пароль с каждым запросом.
\end{enumerate}
\\
\textbf{Необходимо реализовать многопоточную обработку запросов.}
\begin{enumerate}
    \item Для многопоточного чтения запросов использовать создание нового потока (java.lang.Thread)
    \item Для многопотчной обработки полученного запроса использовать ForkJoinPool
    \item Для многопоточной отправки ответа использовать ForkJoinPool
    \item Для синхронизации доступа к коллекции использовать java.util.Collections.synchronizedXXX
\end{enumerate}
\\
\textbf{Порядок выполнения работы:}
\begin{enumerate}
    \item В качестве базы данных использовать PostgreSQL.
    \item Для подключения к БД на кафедральном сервере использовать хост pg, имя базы данных - studs, имя пользователя/пароль совпадают с таковыми для подключения к серверу.
\end{enumerate}
\\
\section{Исходный код}
\url{https://github.com/Kyoto67/VT_labs_1/tree/Programming_lab7}