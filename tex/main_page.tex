%! Author = kyoto
%! Date = 08.03.2022


\section{Задание}
Синтезировать цикл исполнения для выданных преподавателем команд. Разработать тестовые программы, которые проверяют
каждую из синтезированных команд. Загрузить в микропрограммную память БЭВМ циклы исполнения синтезированных команд,
загрузить в основную память БЭВМ тестовые программы. Проверить и отладить разработанные тестовые программы и микропрограммы.\\


\begin{figure}[H]
    \centering
    \includegraphics[scale=0.5]{img/variant}
\end{figure}


\section{Программа}
\begin{center}
    \begin{tabular}{|c|c|c|}
        \hline
        Адрес & Микрокоманда & Действие                              \\
        \hline
        01    & 00A0009004   & IP → BR, AR                           \\
        \hline
        02    & 0104009420   & BR + 1 → IP; MEM(AR) → DR             \\
        \hline
        03    & 0002009001   & DR → CR                               \\
        \hline
        04    & 8109804002   & if CR(15) = 1 then GOTO CHKBR @ 09    \\
        \hline
        05    & 810C404002   & if CR(14) = 1 then GOTO CHKABS @ 0C   \\
        \hline
        06    & 810C204002   & if CR(13) = 1 then GOTO CHKABS @ 0C   \\
        \hline
        07    & 8078104002   & if CR(12) = 0 then GOTO ADDRLESS @ 78 \\
        \hline
        78    & 81A4084002   & if CR(11) = 1 then GOTO AL1XXX @ A4   \\
        \hline
        A4    & 81B5044002   & if CR(10) = 1 then GOTO AL11XX @ B5   \\
        \hline
        B5    & 81BB024002   & if CR(9) = 1 then GOTO AL111X @ BB    \\
        \hline
        BB    & 81E1014002   & if CR(8) = 1 then GOTO RESERVED @ E1  \\
        \hline
        E1    & 80E3801002   & if CR(7) = 0 then GOTO @ E3;          \\
        \hline
        E2    & 8001101040   & GOTO INFETCH @ 01                     \\
        \hline
        E3    & 80E5401002   & if CR(6) = 0 then GOTO @ E5;          \\
        \hline
        E4    & 8001101040   & GOTO INFETCH @ 01                     \\
        \hline
        E5    & 80E7201002   & if CR(5) = 0 then GOTO @ E7;          \\
        \hline
        E6    & 8001101040   & GOTO INFETCH @ 01                     \\
        \hline
        E7    & 80E9101002   & if CR(4) = 0 then GOTO @ E9;          \\
        \hline
        E8    & 8001101040   & GOTO INFETCH @ 01                     \\
        \hline
        E9    & 80EB081002   & if CR(3) = 0 then GOTO @ EB;          \\
        \hline
        EA    & 8001101040   & GOTO INFETCH @ 01                     \\
        \hline
        EB    & 80ED041002   & if CR(2) = 0 then GOTO @ ED;          \\
        \hline
        EC    & 8001101040   & GOTO INFETCH @ 01                     \\
        \hline
        ED    & 80EF021002   & if CR(1) = 0 then GOTO @ EF;          \\
        \hline
        EE    & 8001101040   & GOTO INFETCH @ 01                     \\
        \hline
        EF    & 81F1011002   & if CR(0) = 1 then GOTO @ F1;          \\
        \hline
        F0    & 8001101040   & GOTO INFETCH @ 01                     \\
        \hline
        F1    & 00A0009008   & SP → AR, BR;                         \\
        \hline
        F2    & 0120009420   & BR + 1 → BR; MEM(AR) → DR;          \\
        \hline
        F3    & 0080009020   & BR → AR;                             \\
        \hline
        F4    & 0020009001   & DR → BR;                             \\
        \hline
        F5    & 0100000000   & MEM(AR) → DR;                        \\
        \hline
        F6    & 0080009008   & SP → AR;                             \\
        \hline
        F7    & 0200000000   & DR → MEM(AR);                        \\
        \hline
        F8    & 0001009020   & BR → DR;                             \\
        \hline
        F9    & 0080009408   & SP + 1 → AR;                         \\
        \hline
        FA    & 0200000000   & DR → MEM(AR);                        \\
        \hline
        FB    & 80C4101040   & GOTO INT @ C4                         \\
        \hline
    \end{tabular}
\end{center}


\section{Проверка программы:}
\begin{center}
    \begin{tabular}{c}
        \begin{lstlisting}[basicstyle=\ttfamily\tiny]
            ORG         0x2E1
            X1:         WORD 0x322
            X2:         WORD 0x228
            PS:         WORD 0x018F
            RES1:       WORD 0x0
            RES2:       WORD 0x0
            RESFLAGS:   WORD 0x0
            START:      LD  X1
                        PUSH
                        LD  X2
                        PUSH
                        LD PS
                        PUSH
                        POPF
                        WORD 0F01
                        PUSHF
                        POP
                        CMP PS
                        BEQ FLAGSTRUE
            FLAGSTRUE:  LD (RESFLAGS)+
                        POP
                        CMP X1
                        BEQ X1TRUE
            X1TRUE:     LD (RES1)+
                        POP
                        CMP X2
                        BEQ X2TRUE
            X2TRUE:     LD(RES2)+
                        LD #0x1
                        AND $RES1
                        AND $RES2
                        AND $RESFLAGS
                        ST  RESMAIN
                        HLT
            RESMAIN:    WORD 0x0
        \end{lstlisting}
    \end{tabular}
\end{center}


\small{
\begin{flushleft}
    \begin{tabular}{|c|c|c|c|c|c|c|c|c|c|c|c|c|}
        \hline
        МР до вы- & \multicolumn{12}{1}{Содержимое памяти и регистров процессора после выборки и исполнения микрокоманды} \\
        \cline{2-13}
        борки МК & MR         & Адрес & Значение & IP  & CR   & AR  & DR   & BR   & AC   & SP  & NZVC & МР \\
        \hline
        &            &       &          & 2EE & 0900 & 7FD & 018F & 02ED & 018F & 7FE & 1111 & 01 \\
        \hline
        01       & 00A0009004 &       &          & 2EE & 0900 & 2EE & 018F & 02EE & 018F & 7FE & 1111 & 02 \\
        \hline
        02       & 0104009420 & 2EE   & 0F01(L)  & 2EF & 0900 & 2EE & 0F01 & 02EE & 018F & 7FE & 1111 & 03 \\
        \hline
        03       & 0002009001 &       &          & 2EF & 0F01 & 2EE & 0F01 & 02EE & 018F & 7FE & 1111 & 04 \\
        \hline
        04       & 0002009001 &       &          & 2EF & 0F01 & 2EE & 0F01 & 02EE & 018F & 7FE & 1111 & 05 \\
        \hline
        05       & 0002009001 &       &          & 2EF & 0F01 & 2EE & 0F01 & 02EE & 018F & 7FE & 1111 & 06 \\
        \hline
        06       & 0002009001 &       &          & 2EF & 0F01 & 2EE & 0F01 & 02EE & 018F & 7FE & 1111 & 07 \\
        \hline
        07       & 0002009001 &       &          & 2EF & 0F01 & 2EE & 0F01 & 02EE & 018F & 7FE & 1111 & 78 \\
        \hline
        78       & 0002009001 &       &          & 2EF & 0F01 & 2EE & 0F01 & 02EE & 018F & 7FE & 1111 & A4 \\
        \hline
        A4       & 0002009001 &       &          & 2EF & 0F01 & 2EE & 0F01 & 02EE & 018F & 7FE & 1111 & B5 \\
        \hline
        B5       & 0002009001 &       &          & 2EF & 0F01 & 2EE & 0F01 & 02EE & 018F & 7FE & 1111 & BB \\
        \hline
        BB       & 0002009001 &       &          & 2EF & 0F01 & 2EE & 0F01 & 02EE & 018F & 7FE & 1111 & E1 \\
        \hline
        E1       & 0002009001 &       &          & 2EF & 0F01 & 2EE & 0F01 & 02EE & 018F & 7FE & 1111 & E3 \\
        \hline
        E3       & 0002009001 &       &          & 2EF & 0F01 & 2EE & 0F01 & 02EE & 018F & 7FE & 1111 & E5 \\
        \hline
        E5       & 0002009001 &       &          & 2EF & 0F01 & 2EE & 0F01 & 02EE & 018F & 7FE & 1111 & E7 \\
        \hline
        E9       & 0002009001 &       &          & 2EF & 0F01 & 2EE & 0F01 & 02EE & 018F & 7FE & 1111 & EB \\
        \hline
        EB       & 0002009001 &       &          & 2EF & 0F01 & 2EE & 0F01 & 02EE & 018F & 7FE & 1111 & ED \\
        \hline
        ED       & 0002009001 &       &          & 2EF & 0F01 & 2EE & 0F01 & 02EE & 018F & 7FE & 1111 & EF \\
        \hline
        EF       & 0002009001 &       &          & 2EF & 0F01 & 2EE & 0F01 & 02EE & 018F & 7FE & 1111 & F1 \\
        \hline
        F1       & 00A0009008 &       &          & 2EF & 0F01 & 7FE & 0F01 & 07FE & 018F & 7FE & 1111 & F2 \\
        \hline
        F2       & 0120009420 & 7FE   & 0228(L)  & 2EF & 0F01 & 7FE & 0228 & 07FF & 018F & 7FE & 1111 & F3 \\
        \hline
        F3       & 0080009020 &       &          & 2EF & 0F01 & 7FF & 0228 & 07FF & 018F & 7FE & 1111 & F4 \\
        \hline
        F4       & 0020009001 &       &          & 2EF & 0F01 & 7FF & 0228 & 0228 & 018F & 7FE & 1111 & F5 \\
        \hline
        F5       & 0100000000 & 7FF   & 0322(L)  & 2EF & 0F01 & 7FF & 0322 & 0228 & 018F & 7FE & 1111 & F6 \\
        \hline
        F6       & 0080009008 &       &          & 2EF & 0F01 & 7FE & 0322 & 0228 & 018F & 7FE & 1111 & F7 \\
        \hline
        F7       & 0200000000 & 7FE   & 0322(S)  & 2EF & 0F01 & 7FE & 0322 & 0228 & 018F & 7FE & 1111 & F8 \\
        \hline
        F8       & 0001009020 &       &          & 2EF & 0F01 & 7FE & 0228 & 0228 & 018F & 7FE & 1111 & F9 \\
        \hline
        F9       & 0080009408 &       &          & 2EF & 0F01 & 7FF & 0228 & 0228 & 018F & 7FE & 1111 & FA \\
        \hline
        FA       & 0200000000 & 7FF   & 0228(S)  & 2EF & 0F01 & 7FF & 0228 & 0228 & 018F & 7FE & 1111 & FB \\
        \hline
        FB       & 80C4101040 &    \multicolumn{11}{1}{GO TO INT @C4} \\
        \hline

    \end{tabular}
\end{flushleft}
}
%
%\section{Методика проверки.}
%Проверка обработки прерываний:
%\begin{enumerate}
%    \item Загрузить текст программы в БЭВМ во вкладку Assembler.
%    \item Заменить NOP в тексте программы на HLT.
%    \item Запустить программу в режиме РАБОТА.
%    \item Установить «Готовность ВУ-1».
%    \item Дождаться остановки программы.
%    \item Записать текущее значение X из памяти БЭВМ:
%    \begin{enumerate}
%        \item Запомнить текущее состояние счетчика команд.
%        \item Ввести в клавишный регистр значение 0x02E
%        \item Нажать «Ввод адреса».
%        \item Нажать «Чтение».
%        \item Записать значение регистра данных.
%        \item Вернуть счетчик команд в исходное состояние.
%    \end{enumerate}
%    \item  Записать результат обработки прерывания – содержимое DR контроллера ВУ-1
%    \item Рассчитать ожидаемое значение обработки прерывания по формуле $3*X-5$, записать его
%    \item Сверить ожидаемые данные с полученными.
%    \item Нажать «Продолжение».
%    \item Ввести в ВУ-2 произвольный набор нулей и единиц, записать его
%    \item Установить «Готовность ВУ-2».
%    \item Дождаться остановки программы.
%    \item Записать текущее значение X из памяти БЭВМ (аналогично п.6).
%    \item Нажать «Продолжение».
%    \item Записать текущее значение X из памяти БЭВМ (аналогично п.6).
%    \item Рассчитать ожидаемое значение переменной X после обработки прерывания \\
%    (Если X будет меньше минимального из ОДЗ, то запишется нижняя граница.)
%    \item Проверить правильность маскирования полученного числа последними\\
%    Четырьмя битами из введённого набора на ВУ.
%\end{enumerate}
%Проверка основной программы:
%\begin{enumerate}
%    \item Загрузить текст программы в БЭВМ.
%    \item Записать в переменную X значение, меньшее минимального доступного по ОДЗ (X < -41)
%    \begin{enumerate}
%        \item Записать значение IP.
%        \item Ввести адрес 0х2Е.
%        \item Ввести желаемое значение в Input Register.
%        \item Нажать "Запись".
%        \item Вернуть в IR исходный IP.
%        \item Нажать "Ввод адреса".
%    \end{enumerate}
%    \item Запустить программу в режиме попрограммного выполнения.
%    \item Убедиться, что при первой итерации в X записывается минимальное значение по ОДЗ (FFD7) (аналогично п.6 из предыдущей инструкции)
%    \item Пройти нужное количество шагов программы, убедиться, что при увеличении X на 1, до того момента, когда он равен 45, происходит сброс значения в
%    минимальное по ОДЗ (FFD7).
%\end{enumerate}
%
%Проверка обработки прерываний с ВУ-1:
%\begin{center}
%    \begin{tabular}{|c|c|c|c|}
%        \hline
%        X   & Предполагаемый результат & Полученный результат & Верно? \\
%        \hline
%        0x4 & $3*4-5=7$                & 0x7                  & да.    \\
%        \hline
%    \end{tabular}
%\end{center}
%Проверка обработки прерываний с ВУ-2:
%\begin{center}
%    \begin{tabular}{|c|c|c|c|c|}
%        \hline
%        Введённые данные с ВУ-2 & $X_{orig}$ & Предполагаемый результат & $X_{res}$ & Верно? \\
%        \hline
%        1010 0101               & 0xB (1011) & 1 (0001)                 & 1(0001)   & да.    \\
%        \hline
%    \end{tabular}
%\end{center}
%Проверка работы основной программы (случай выхода из ОДЗ):
%\begin{center}
%    \begin{tabular}{|c|c|c|c|}
%        \hline
%        Значение икса перед & Какое значение должно  & Какой получился & Верно? \\
%        выходом из ОДЗ      & было быть без проверки &                 &        \\
%        \hline
%        0x2C                & 0x2D                   & 0xFFD7          & да.    \\
%        0xFFD0              & 0xFFD1                 & 0xFFD7          & да.    \\
%        \hline
%    \end{tabular}
%\end{center}
