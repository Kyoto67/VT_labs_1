%! Author = kyoto
%! Date = 08.03.2022


\section{Задание}
Разделить программу из лабораторной работы №5 на клиентский и серверный модули. Серверный модуль должен осуществлять выполнение команд по управлению коллекцией. Клиентский модуль должен в интерактивном режиме считывать команды, передавать их для выполнения на сервер и выводить результаты выполнения.

\textbf{Необходимо выполнить следующие требования:}

\noindentОперации обработки объектов коллекции должны быть реализованы с помощью Stream API с использованием лямбда-выражений.\\
Объекты между клиентом и сервером должны передаваться в сериализованном виде.\\
Объекты в коллекции, передаваемой клиенту, должны быть отсортированы по названию\\
Клиент должен корректно обрабатывать временную недоступность сервера.\\
Обмен данными между клиентом и сервером должен осуществляться по протоколу TCP\\
Для обмена данными на сервере необходимо использовать потоки ввода-вывода\\
Для обмена данными на клиенте необходимо использовать сетевой канал\\
Сетевые каналы должны использоваться в неблокирующем режиме.\\
\\
\subsection{Обязанности серверного приложения:}

\noindentРабота с файлом, хранящим коллекцию.\\
Управление коллекцией объектов.\\
Назначение автоматически генерируемых полей объектов в коллекции.\\
Ожидание подключений и запросов от клиента.\\
Обработка полученных запросов (команд).\\
Сохранение коллекции в файл при завершении работы приложения.\\
Сохранение коллекции в файл при исполнении специальной команды, доступной только серверу (клиент такую команду отправить не может).\\
Серверное приложение должно состоять из следующих модулей (реализованных в виде одного или нескольких классов):\\
Модуль приёма подключений.\\
Модуль чтения запроса.\\
Модуль обработки полученных команд.\\
Модуль отправки ответов клиенту.\\
Сервер должен работать в однопоточном режиме.\\
\\
\textbf{Серверное приложение должно состоять из следующих модулей (реализованных в виде одного или нескольких классов):\\}
Модуль приёма подключений.\\
Модуль чтения запроса.\\
Модуль обработки полученных команд.\\
Модуль отправки ответов клиенту.\\
\\
\subsection{Обязанности клиентского приложения:}

\noindentЧтение команд из консоли.\\
Валидация вводимых данных.\\
Сериализация введённой команды и её аргументов.\\
Отправка полученной команды и её аргументов на сервер.\\
Обработка ответа от сервера (вывод результата исполнения команды в консоль).\\
Команду save из клиентского приложения необходимо убрать.\\
Команда exit завершает работу клиентского приложения.\\
Важно! Команды и их аргументы должны представлять из себя объекты классов. Недопустим обмен "простыми" строками. Так, для команды add или её аналога необходимо сформировать объект, содержащий тип команды и объект, который должен храниться в вашей коллекции.\\
\\
\section{Исходный код}
\url{https://github.com/Kyoto67/VT_labs_1/tree/Programming_lab6}