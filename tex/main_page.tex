%! Author = kyoto
%! Date = 08.03.2022


\section{Задание}
Синтезировать цикл исполнения для выданных преподавателем команд. Разработать тестовые программы, которые проверяют
каждую из синтезированных команд. Загрузить в микропрограммную память БЭВМ циклы исполнения синтезированных команд,
загрузить в основную память БЭВМ тестовые программы. Проверить и отладить разработанные тестовые программы и микропрограммы.\\


\begin{figure}[H]
    \centering
    \includegraphics[scale=0.5]{img/variant}
\end{figure}


\section{Программа}
\begin{center}
    \begin{tabular}{|c|c|c|}
        \hline
        Адрес & Микрокоманда & Действие                              \\
        \hline
        01    & 00A0009004   & IP → BR, AR                           \\
        \hline
        02    & 0104009420   & BR + 1 → IP; MEM(AR) → DR             \\
        \hline
        03    & 0002009001   & DR → CR                               \\
        \hline
        04    & 8109804002   & if CR(15) = 1 then GOTO CHKBR @ 09    \\
        \hline
        05    & 810C404002   & if CR(14) = 1 then GOTO CHKABS @ 0C   \\
        \hline
        06    & 810C204002   & if CR(13) = 1 then GOTO CHKABS @ 0C   \\
        \hline
        07    & 8078104002   & if CR(12) = 0 then GOTO ADDRLESS @ 78 \\
        \hline
        78    & 81A4084002   & if CR(11) = 1 then GOTO AL1XXX @ A4   \\
        \hline
        A4    & 81B5044002   & if CR(10) = 1 then GOTO AL11XX @ B5   \\
        \hline
        B5    & 81BB024002   & if CR(9) = 1 then GOTO AL111X @ BB    \\
        \hline
        BB    & 81E1014002   & if CR(8) = 1 then GOTO RESERVED @ E1  \\
        \hline
        E0    & 80FC101040   & GOTO RESERVED @FC                     \\
        \hline
        E1    & 80E3801002   & if CR(7) = 0 then GOTO @ E3;          \\
        \hline
        E2    & 8001101040   & GOTO INFETCH @ 01                     \\
        \hline
        E3    & 80E5401002   & if CR(6) = 0 then GOTO @ E5;          \\
        \hline
        E4    & 8001101040   & GOTO INFETCH @ 01                     \\
        \hline
        E5    & 80E7201002   & if CR(5) = 0 then GOTO @ E7;          \\
        \hline
        E6    & 8001101040   & GOTO INFETCH @ 01                     \\
        \hline
        E7    & 80E9101002   & if CR(4) = 0 then GOTO @ E9;          \\
        \hline
        E8    & 8001101040   & GOTO INFETCH @ 01                     \\
        \hline
        E9    & 80EB081002   & if CR(3) = 0 then GOTO @ EB;          \\
        \hline
        EA    & 8001101040   & GOTO INFETCH @ 01                     \\
        \hline
        EB    & 80ED041002   & if CR(2) = 0 then GOTO @ ED;          \\
        \hline
        EC    & 8001101040   & GOTO INFETCH @ 01                     \\
        \hline
        ED    & 80EF021002   & if CR(1) = 0 then GOTO @ EF;          \\
        \hline
        EE    & 8001101040   & GOTO INFETCH @ 01                     \\
        \hline
        EF    & 81F1011002   & if CR(0) = 1 then GOTO @ F1;          \\
        \hline
        F0    & 8001101040   & GOTO INFETCH @ 01                     \\
        \hline
        F1    & 00A0009008   & SP → AR, BR;                          \\
        \hline
        F2    & 0120009420   & BR + 1 → BR; MEM(AR) → DR;            \\
        \hline
        F3    & 0080009020   & BR → AR;                              \\
        \hline
        F4    & 0020009001   & DR → BR;                              \\
        \hline
        F5    & 0100000000   & MEM(AR) → DR;                         \\
        \hline
        F6    & 0080009008   & SP → AR;                              \\
        \hline
        F7    & 0200000000   & DR → MEM(AR);                         \\
        \hline
        F8    & 0001009020   & BR → DR;                              \\
        \hline
        F9    & 0080009408   & SP + 1 → AR;                          \\
        \hline
        FA    & 0200000000   & DR → MEM(AR);                         \\
        \hline
        FB    & 80C4101040   & GOTO INT @ C4                         \\
        \hline
    \end{tabular}
\end{center}


\section{Проверка программы:}
\begin{center}
    \begin{tabular}{c}
        \begin{lstlisting}[basicstyle=\ttfamily\tiny]
            ORG         0x2E1
            X1:         WORD 0x322
            X2:         WORD 0x228
            PS:         WORD 0x018F
            RES1:       WORD 0x0
            RES2:       WORD 0x0
            RESFLAGS:   WORD 0x0
            START:      LD  X1
                        PUSH
                        LD  X2
                        PUSH
                        LD PS
                        PUSH
                        POPF
                        WORD 0x0F01
                        PUSHF
                        POP
                        CMP PS
                        BNE NEXT1
                        LD (RESFLAGS)+
                NEXT1:  POP
                        CMP X1
                        BNE NEXT2
                        LD (RES1)+
                NEXT2:  POP
                        CMP X2
                        BNE NEXT3
                        LD(RES2)+
                NEXT3:  LD #0x1
                        AND $RES1
                        AND $RES2
                        AND $RESFLAGS
                        ST  RESMAIN
                        HLT
            RESMAIN:    WORD 0x0
        \end{lstlisting}
    \end{tabular}
\end{center}

\section{Трассировка команды.}

\small{
    \begin{flushleft}
        \begin{tabular}{|c|c|c|c|c|c|c|c|c|c|c|c|c|}
            \hline
            МР до вы- & \multicolumn{12}{1}{Содержимое памяти и регистров процессора после выборки и исполнения микрокоманды} \\
            \cline{2-13}
            борки МК & MR         & Адрес & Значение & IP  & CR   & AR  & DR   & BR   & AC   & SP  & NZVC & МР \\
            \hline
            &            &       &          & 2EE & 0900 & 7FD & 018F & 02ED & 018F & 7FE & 1111 & 01 \\
            \hline
            01       & 00A0009004 &       &          & 2EE & 0900 & 2EE & 018F & 02EE & 018F & 7FE & 1111 & 02 \\
            \hline
            02       & 0104009420 & 2EE   & 0F01(L)  & 2EF & 0900 & 2EE & 0F01 & 02EE & 018F & 7FE & 1111 & 03 \\
            \hline
            03       & 0002009001 &       &          & 2EF & 0F01 & 2EE & 0F01 & 02EE & 018F & 7FE & 1111 & 04 \\
            \hline
            04       & 0002009001 &       &          & 2EF & 0F01 & 2EE & 0F01 & 02EE & 018F & 7FE & 1111 & 05 \\
            \hline
            05       & 0002009001 &       &          & 2EF & 0F01 & 2EE & 0F01 & 02EE & 018F & 7FE & 1111 & 06 \\
            \hline
            06       & 0002009001 &       &          & 2EF & 0F01 & 2EE & 0F01 & 02EE & 018F & 7FE & 1111 & 07 \\
            \hline
            07       & 0002009001 &       &          & 2EF & 0F01 & 2EE & 0F01 & 02EE & 018F & 7FE & 1111 & 78 \\
            \hline
            78       & 0002009001 &       &          & 2EF & 0F01 & 2EE & 0F01 & 02EE & 018F & 7FE & 1111 & A4 \\
            \hline
            A4       & 0002009001 &       &          & 2EF & 0F01 & 2EE & 0F01 & 02EE & 018F & 7FE & 1111 & B5 \\
            \hline
            B5       & 0002009001 &       &          & 2EF & 0F01 & 2EE & 0F01 & 02EE & 018F & 7FE & 1111 & BB \\
            \hline
            BB       & 0002009001 &       &          & 2EF & 0F01 & 2EE & 0F01 & 02EE & 018F & 7FE & 1111 & E1 \\
            \hline
            E1       & 0002009001 &       &          & 2EF & 0F01 & 2EE & 0F01 & 02EE & 018F & 7FE & 1111 & E3 \\
            \hline
            E3       & 0002009001 &       &          & 2EF & 0F01 & 2EE & 0F01 & 02EE & 018F & 7FE & 1111 & E5 \\
            \hline
            E5       & 0002009001 &       &          & 2EF & 0F01 & 2EE & 0F01 & 02EE & 018F & 7FE & 1111 & E7 \\
            \hline
            E9       & 0002009001 &       &          & 2EF & 0F01 & 2EE & 0F01 & 02EE & 018F & 7FE & 1111 & EB \\
            \hline
            EB       & 0002009001 &       &          & 2EF & 0F01 & 2EE & 0F01 & 02EE & 018F & 7FE & 1111 & ED \\
            \hline
            ED       & 0002009001 &       &          & 2EF & 0F01 & 2EE & 0F01 & 02EE & 018F & 7FE & 1111 & EF \\
            \hline
            EF       & 0002009001 &       &          & 2EF & 0F01 & 2EE & 0F01 & 02EE & 018F & 7FE & 1111 & F1 \\
            \hline
            F1       & 00A0009008 &       &          & 2EF & 0F01 & 7FE & 0F01 & 07FE & 018F & 7FE & 1111 & F2 \\
            \hline
            F2       & 0120009420 & 7FE   & 0228(L)  & 2EF & 0F01 & 7FE & 0228 & 07FF & 018F & 7FE & 1111 & F3 \\
            \hline
            F3       & 0080009020 &       &          & 2EF & 0F01 & 7FF & 0228 & 07FF & 018F & 7FE & 1111 & F4 \\
            \hline
            F4       & 0020009001 &       &          & 2EF & 0F01 & 7FF & 0228 & 0228 & 018F & 7FE & 1111 & F5 \\
            \hline
            F5       & 0100000000 & 7FF   & 0322(L)  & 2EF & 0F01 & 7FF & 0322 & 0228 & 018F & 7FE & 1111 & F6 \\
            \hline
            F6       & 0080009008 &       &          & 2EF & 0F01 & 7FE & 0322 & 0228 & 018F & 7FE & 1111 & F7 \\
            \hline
            F7       & 0200000000 & 7FE   & 0322(S)  & 2EF & 0F01 & 7FE & 0322 & 0228 & 018F & 7FE & 1111 & F8 \\
            \hline
            F8       & 0001009020 &       &          & 2EF & 0F01 & 7FE & 0228 & 0228 & 018F & 7FE & 1111 & F9 \\
            \hline
            F9       & 0080009408 &       &          & 2EF & 0F01 & 7FF & 0228 & 0228 & 018F & 7FE & 1111 & FA \\
            \hline
            FA       & 0200000000 & 7FF   & 0228(S)  & 2EF & 0F01 & 7FF & 0228 & 0228 & 018F & 7FE & 1111 & FB \\
            \hline
            FB & 80C4101040 & \multicolumn{11}{1}{GO TO INT @C4} \\
            \hline

        \end{tabular}
    \end{flushleft}
}
\newpage


\section{Методика проверки.}

\begin{enumerate}
    \item Открыть терминал.
    \item Ввести \texttt{java -Dmode=cli -jar bcomp-ng14507.jar} , нажать ENTER.
    \item В открывшейся программе ввести:   \\
    \texttt{ma bb   \\
    ma bb   \\
    mw 81E1014002   \\
    ma e0           \\
    ma e0           \\
    mw 80FC101040   \\
    mw 80E3801002   \\
    mw 8001101040   \\
    mw 80E5401002   \\
    mw 8001101040   \\
    mw 80E7201002   \\
    mw 8001101040   \\
    mw 80E9101002   \\
    mw 8001101040   \\
    mw 80EB081002   \\
    mw 8001101040   \\
    mw 80ED041002   \\
    mw 8001101040   \\
    mw 80EF021002   \\
    mw 8001101040   \\
    mw 81F1011002   \\
    mw 8001101040   \\
    mw 00A0009008   \\
    mw 0120009420   \\
    mw 0080009020   \\
    mw 0020009001   \\
    mw 0100000000   \\
    mw 0080009008   \\
    mw 0200000000   \\
    mw 0001009020   \\
    mw 0080009408   \\
    mw 0200000000   \\
    mw 80C4101040}
    \item Ввести в терминал \texttt{asm} и нажать ENTER.
    \item Ввести в терминал тестовую программу:\\
    \texttt{ORG 0x2E1   \\
    X1: WORD 0x322  \\
    X2: WORD 0x228  \\
    PS: WORD 0x018F \\
    RES1: WORD 0x0    \\
    RES2: WORD 0x0    \\
    RESFLAGS: WORD 0x0    \\
    START: LD X1      \\
    PUSH        \\
    LD X2      \\
    PUSH        \\
    LD PS       \\
    PUSH        \\
    POPF        \\
    WORD 0x0F01   \\
    PUSHF       \\
    POP         \\
    CMP PS      \\
    BNE NEXT1   \\
    LD (RESFLAGS)+  \\
    NEXT1: POP         \\
    CMP X1      \\
    BNE NEXT2   \\
    LD (RES1)+  \\
    NEXT2: POP         \\
    CMP X2      \\
    BNE NEXT3   \\
    LD (RES2)+   \\
    NEXT3: LD #0x1     \\
    AND \$RES1  \\
    AND \$RES2  \\
    AND \$RESFLAGS  \\
    ST RESMAIN \\
    HLT         \\
    RESMAIN: WORD 0x0    \\
    END}
    \item Ввести \texttt{ru} (режим Работа) и нажать ENTER.
    \item Ввести \texttt{s} (старт программы) и нажать ENTER.
    \item Проверить результат в ячейке 302 (в терминате будет выведен результат в формате: \texttt{ЯЧЕЙКА ЗНАЧЕНИЕ})
    \item Если итоговый результат тест не равен единице, то проверить ячейки 2E4, 2E5, 2E6:\\
    2E4 - Результат того, что второе по счёту число переместилось на вершину стека. \\
    2E5 - Результат того, что число с вершины стека переместилось на второе место.  \\
    2E6 - Результат того, что флаги NZVC остались неизменны.
    \item Спок.
\end{enumerate}
